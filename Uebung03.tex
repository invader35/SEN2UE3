\documentclass[10pt,a4paper, parskip=half]{scrartcl}
\usepackage[utf8x]{inputenc}
\usepackage[german]{babel}
\usepackage{ucs}
\usepackage{amsmath}
\usepackage{amsfonts}
\usepackage{amssymb}
\usepackage{graphicx}
\usepackage{color}
\usepackage{listings}
\usepackage{hyperref}
\usepackage[superscript]{cite}
\author{Schett Matthias}
\title{SEN-Übung 2.3}
\begin{document}
\definecolor{gray}{rgb}{0.9,0.9,0.9}
\lstset{language=[Visual]Basic, morekeywords={param, local}}
\maketitle

\section{Aufgabe 1}
\label{sec:Aufgabe1}

\subsection{L\"{o}sungsidee}
Es soll ein Programm erstellt werden dass Testdaten nach folgender Grammatik ausgibt:

\lstinputlisting[breaklines=true, tabsize=4, showstringspaces=false,backgroundcolor=\color{gray}]
{Grammar.txt}

Die einzelnen Daten sollen mittels Zufallszahlengenerator ermittelt werden.
Zu Beginn wird der Dateikopf geschrieben, die übernimmt die interne writeFileHeader() Funktion.
Danach werden die einzelnen Werte ermittelt und ebenfalls in die Datei geschrieben
Die Angaben zum Dateinamen und der Anzahl der zu generierenden Testdaten, wird mittels Kommandozeilenparamenter übergeben.

\paragraph{Beispiel} % (fold)
\label{par:Beispiel}
TestDataGenerator.exe TestFile1.txt 300
\newline
Generiert ein TestFile1.txt mit 300 Testwerten.

\subsection{Programmcode in C++}

Der Programmcode Aufgabe befindet sich im Anhang

\subsection{Testf\"{a}lle}
Als Testfälle werden 3 Dateien erzeugt TestFile1.txt mit 10, TestFile2.txt mit 20 und TestFile3.txt mit 24 Einträgen.
Der Aufruf sieht daher so aus:
\newline
\indent TestDataCreator.exe TestFile1.txt 10 TestFile2.txt 20 TestFile3.txt 24

\newpage
\section{Aufgabe 2}

\subsection{L\"{o}sungsidee}
\nameref{sec:Aufgabe1} soll nun so erweitert werden, dass die Grammatik der erstellten Dateien geprüft wird und eine Schnittmenge aller 
Test Daten in eine Result.txt geschrieben werden.
Dafür gibt es nun die Methoden checkGrammar().
checkGrammar() prüft zuerst den Dateikopf und anschließend den Rest der Datei bei einem Fehler wird \"unkown file format\" und die Stelle
an der sich der Fehler befindet ausgegeben.

Anschließend werden unter zuhilfe Name von std::set(siehe \ref{subsubsec:Set}) alle Werte der Datei gepseichert. 

Das wird mit allen Dateien wiederholt.
Anschließend wird über den Inhalt der std::set Datenstruktur iteriert und jeder Eintrag in die Result.txt geschrieben.

\subsubsection{Informationen zu std::set}
\label{subsubsec:Set}

Die Datenstruktur std::set ist eine dynamische Datenstruktur ähnlich der einzel und doppelt verketteten Listen.
Alle in ihr befindlichen Einträgen sind aufsteigend sortiert.
\newline Weiters sind alle Einträge in einem std::set einzigartig, dieser Umstand wird bei jedem insert() geprüft, dadurch eignet sie sich
besonders für diese Aufgabe.

\subsection{Programmcode in C++}

Der Programmcode befindet sich im Anhang

\subsection{Testf\"{a}lle}
Die Testfälle sind indentisch mit denen aus \nameref{sec:Aufgabe1}

\newpage
\appendix
\section{Programmcode in C++}

\subsection{CheckGrammar Modul Header}
\lstinputlisting[breaklines=true, tabsize=4, showstringspaces=false,backgroundcolor=\color{gray}, numbers=left]
{TestDataCreator/GrammarCheck.h}

\subsection{CheckGrammar Modul Implmentierung}
\lstinputlisting[breaklines=true, tabsize=4, showstringspaces=false,backgroundcolor=\color{gray}, numbers=left]
{TestDataCreator/GrammarCheck.cpp}

\newpage
\subsection{TestDataCreator Modul Header}
\lstinputlisting[breaklines=true, tabsize=4, showstringspaces=false,backgroundcolor=\color{gray}, numbers=left]
{TestDataCreator/TestDataCreator.h}

\subsection{TestDataCreator Modul Implmentierung}
\lstinputlisting[breaklines=true, tabsize=4, showstringspaces=false,backgroundcolor=\color{gray}, numbers=left]
{TestDataCreator/TestDataCreator.cpp}

\newpage
\subsection{Testreiber}
\lstinputlisting[breaklines=true, tabsize=4, showstringspaces=false,backgroundcolor=\color{gray}, numbers=left]
{TestDataCreator/Main.cpp}

\end{document}